\documentclass{article}
\usepackage[utf8]{inputenc}
\usepackage{graphicx}
\usepackage{apacite}
\graphicspath{ {images/} }
\usepackage{blindtext}
\begin{document}
\begin{titlepage}
{\includegraphics[width=0.2\textwidth]{udea.png}\par}
\vspace{1cm}
\centering
{\bfseries\LARGE Universidad de Antioquia \par}
\vspace{1cm}
{\scshape\Large Facultad de Ingenieria \par}
\vspace{3cm}
{\scshape\Huge Nociones de la memoria del computador \par}
\vspace{3cm}
\vfill
{\Large Autor: \par}
{\Large Arnel David Bravo Tobon  \par}
\vfill
{\Large septiembre 2020 \par}
\end{titlepage}

\begin{abstract}
    Este documento esta basado en el artículo nociones de la memoria de un computador de Augusto Salazar, se hablará de los tipos de memorias que podemos encontrar en un computador, podremos ver que hay memorias temporales las cuales, tienen gran importancia para que los procesos se realicen de una manera más eficiente , de solo lectura donde se encuentran las instrucciones de la máquina y se compone por la BIOS y el setup que son esenciales para el funcionamiento del computador y las memorias para almacenar datos que son importantes en nuestra vida cotidiana. 
\end{abstract}
\section*{Introduccion}
La memoria es muy importante ya que Se utiliza para almacenar datos e instrucciones y es muy necesaria para el correcto funcionamiento de la máquina, existen diferentes tipos de memorias en un computador y cada una cumple con una función, algunas almacenan información, como otras que solo son temporales o de lectura.
\section*{Solucion del cuestionario}
\subsection*{1) Defina que es la memoria del computador.} 
La memoria de la computadora es el espacio de almacenamiento de información   en la computadora, donde se procesan los datos y se almacenan las instrucciones necesarias para su procesamiento.
\cite{aguilera2015modulo}


 \subsection*{2) Mencione los tipos de memoria que conoce y haga una pequeña descripción de cada tipo.}
Memoria portatiles USB:
Es un tipo de dispositivo de almacenamiento de datos que utiliza circuitos de estado sólido para guardar datos e información, y es muy funcional ya que permite extraer e introducir información a un computador.\cite{barrero2005sistema}\\[0.1cm]


Memoria RAM(Memoria de Acceso Aleatorio):
Es uno de los componentes mas importantes de cualquier equipo de computo, Su función principal es almacenar datos e instrucciones para que puedan ser accedidos por otros componentes básicos, de manera que evita que tengan que volver a pasar por el procesador o incluso por la tarjeta gráfica.\\[0.1cm]

Memoria ROM(Memoria de solo lectura):
Es el medio de almacenamiento de programas o datos que permiten el buen funcionamiento de los ordenadores o dispositivos electrónicos a través de la lectura de la información sin que pueda ser destruida o reprogramable.\\[0.1cm]

Disco Duro:
Es un dispositivo de almacenamiento de datos que emplea un sistema de grabación magnética para almacenar y recuperar archivos digitales.

\subsection*{3) Describa como se gestiona la memoria de un computador.}
De la gestión en el computador se encarga el procesador en cierta parte, aunque mayormente es gracias al administrador de memoria presente en los sistemas operativos y su labor consiste en llevar un registro de las partes de memoria que se estén utilizando y aquellas que no, con el fin de asignar espacio en memoria a los procesos cuando éstos la necesiten y liberándola cuando terminen. Así mismo sabe que información necesita ser cargada en RAM y cual no, y lo mismo para los otros tipos de memorias presentes como cache, vram, etc.

¿para qué sirve?
Para optimizar el espacio y poder cargar o intercambiar los programas que van a hacer ejecutados del disco duro a la memoria principal. Llevar un registro de las partes de la memoria que están en uso y de las que no. Si detecta que hay una parte que ya no está en uso, la libera para poder asignarla a los procesos que la necesiten. Facilitar un espacio de memoria para cada proceso y controlar que ninguno de ellos trabaje en zonas de memoria que no le han sido asignados. A el intercambio entre la memoria principal y el disco en los casos en los que la memoria principal no le pueda dar capacidad a todos los procesos que tienen necesidad de ella.\\[0.1cm]


\subsection*{4)¿Qué hace que una memoria sea más rápida que otra? ¿Por qué esto es importante?}

Lo que mide que una memoria sea más rápida que otra es su latencia es decir el tiempo que transcurre entre una petición y su respuesta debido a que una memoria con una latencia baja va a ocasionar que el dispositivo reaccione más rápido, a diferencia cuando la memoria tiene una latencia alta en donde el dispositivo se demorara más tiempo en Reaccionar.(tomado de articulo como funciona las memorias de un computador, Augusto salazar)\\[0.1cm]

La importancia de que una memoria sea rápida es que permite realizar transferencias de información en menos tiempo. Las operaciones de almacenar, borrar y Re almacenar nueva información y datos se completarán más rápidamente, lo que en algunos casos puede marcar una diferencia importante de rendimiento.

\bibliographystyle{apacite}
\bibliography{bibliografia.bib}







\end{document}
